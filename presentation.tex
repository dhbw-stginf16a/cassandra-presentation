\documentclass[
  10pt
%, handout
]{beamer}

\usepackage{pgfpages}

%\setbeameroption{show notes on second screen}

\usetheme{metropolis}
\usepackage{appendixnumberbeamer}

\usepackage{booktabs}
\usepackage[scale=2]{ccicons}

\usepackage{pgfplots}
\usepgfplotslibrary{dateplot}

\usepackage{xspace}
\newcommand{\themename}{\textbf{\textsc{metropolis}}\xspace}

\usepackage{graphicx}
\usepackage{listings}

\usepackage{lmodern}

\usepackage{eurosym}
\usepackage{amsmath, amssymb}
\usepackage[binary-units=true]{siunitx}
\DeclareSIUnit{\EUR}{\text{\euro}}

\usepackage{xcolor}
\newcommand\crule[3][black]{\textcolor{#1}{\rule{#2}{#3}}}
\definecolor{aswe-reactive}{cmyk}{1,0.9,0,0}
\definecolor{aswe-proactive}{cmyk}{0.6,0.9,0,0}
\definecolor{aswe-preferences}{cmyk}{0,0.75,1,0}
\definecolor{aswe-data}{cmyk}{0.85,0.1,1,0}

\setcounter{tocdepth}{1}  % Hide subsections in table of contents

% Content

\title{Introduction to Apache Cassandra}
\subtitle{}
\date{April 10, 2019}
\author{David Marchi, Daniel Schäfer, Erik Zeiske}
% \titlegraphic{\hfill\includegraphics[height=1.5cm]{logo.pdf}}

\begin{document}

\maketitle

\begin{frame}{Agenda}
  \tableofcontents[pausesections]
\end{frame}

\section{Properties (overview)}  % David

\section{Wide Column Store}  % David

\section{Usecases Cassandra is (not) suited for}  % David
\begin{frame}{Pros and cons}
  \only<1->{
    Use if you want/need
    \begin{itemize}
      \item Fast writes (high throughput, not latency)
      \item High availability
      \item Easy (linear) horizontal scalability
      \item No master $\rightarrow$ Any node can be read from and written to
      \item Flexible schema (rows can have missing columns)
      \item Globally distributed cluster
    \end{itemize}
    \note[item]{Writes include \lstinline{INSERT} \lstinline{UPDATE} \lstinline{DELETE}}
    \note[item]{Inter-node communication doesn't increase with more nodes}
    \note[item]{Missing columns aren't even saved to disk (sparse)}
  }

  \only<2->{
    Don't use if you want/need
    \begin{itemize}
      \item Single system instance
      \item Ever changing queries
      \item Lots of updates and deletes interspersed with reads
      \item Transactions (ACID)
      \item Relations (joins, ...)
      \item Column aggregation (\lstinline{GROUP BY})
      \item \lstinline{AUTO INCREMENT}
    \end{itemize}
    \note[item]{Cassandra is \textbf{NOT} column oriented}
  }
\end{frame}

\begin{frame}{Suitable usecases}
  % TODO: Improve this slide - separate abstract and concrete usecases from eachother
  Concrete use-cases
  \begin{itemize}
    \item Recording of financial transaction
    \item Time series data (logs of any kind, e.g. sensors with huge outputs)
    \item Tracking status
    \item Anything append only
  \end{itemize}
\end{frame}

\section{How to use CQL}  % Erik
\subsection{Compare to API (key-value)}
\subsection{Compare to SQL}

\section{How to model data (or rather tables)}  % David

\subsection{Secondary index}  % Daniel

\begin{frame}[fragile]{Secondary index - Why}
  Make it possible to filter by a non primary-key column
  \begin{semiverbatim}
  \only<1->{
  cqlsh:db> CREATE TABLE users (
  cqlsh:db>  username varchar PRIMARY KEY,
  cqlsh:db>  age int
  cqlsh:db> );
  cqlsh:db> SELECT * FROM users WHERE age = 5;
  } \only<2->{
  cqlsh:db> \textcolor{red}{InvalidRequest: [might filter data]}
  } \only<3->{
  cqlsh:db> CREATE INDEX ON users (age);
  cqlsh:db> SELECT * FROM users WHERE age = 5;

   username | age
  ----------+-----
      timmy |   5
  }
  \end{semiverbatim}
  \note<1-2>[item]{Unpredicatable performance \rightarrow ALLOW FILTERING}
  \note<1-2>[item]{Has to go through all nodes and all of their data}
\end{frame}

\begin{frame}{Secondary index - Implementation}
  \includegraphics[width=1.0\textwidth]{resources/distributed_index.png}
  \note[item]{NOT tree, regular table}
  \note[item]{Distributed, same location as data}
  \note[item]{Restrict partition key \rightarrow Find node}
  \note[item]{Heuristic to fetch data from likely nodes in stages}
\end{frame}

\begin{frame}{Secondary index - Summary}
  \begin{itemize}
    \item<1-> Distributed table with different primary key
    \item<2-> Located on same node as indexed data
      \begin{itemize}
        \item Fast to update
        \item Paritions as balanced as data
      \end{itemize}
    \item<3-> Good practice to restrict partition key as well
    \item<4-> Uses heuristic to fetch data from likely nodes $\rightarrow$ \textbf{Use LIMIT}
  \end{itemize}
  \note<3>[item]{Partition key \rightarrow only needs to search single node}
\end{frame}

\begin{frame}{Secondary index - When not to use}
  Avoid columns
  \begin{itemize}
    \item<1-> with very high cardinality (e.g. user id)
    \item<2-> with very low cardinality (e.g. country name)
    \item<3-> that are updated/deleted very frequently
  \end{itemize}
  \note<1>[item]{Low chance of finding - no partition key \rightarrow queries all nodes}
  \note<2>[item]{Results in huge index rows}  % TODO: Think about a good explanation
  \note<3>[item]{Updates have to update two tables}
  \note<3>[item]{Tombstones are stored in index - hard limit of 100K \rightarrow Query FAILS}
\end{frame}

\section{Local reads and writes}  % Erik
\subsection{Write process}
\subsection{Read process}
\subsection{How data is stored on disk}

\section{How the cluster works}  % Daniel
\subsection{What happens when a new node is added}
\subsection{How the cluster is initially set up}

\section{Distributed writes and reads}  % Daniel
\subsection{How data is replicated}
\subsection{How data is read with consensus}

\section{Set up and must knows}  % Daniel

\begin{frame}{Setup - overview}
  \begin{itemize}
    \item<1-> Get enough nodes ($\geq3$ with $8+$ cores and 32GB RAM, lots of disk)
    % http://cassandra.apache.org/doc/latest/operating/hardware.html
    \item<2-> Set up network (each node with own IP reachable from the others)
    \item<3-> Adapt configuration files
    \item<4-> Open firewall for necessary ports
    \begin{itemize}
      \item 7000: node $\leftrightarrow$ node (\lstinline{storage_port})
      \item 9042: client $\leftrightarrow$ node (\lstinline{native_transport_port})
    \end{itemize}
    \item<5-> Start seed nodes and then other nodes
    \item<6-> Use \lstinline{nodetool} or \lstinline{cqlsh} to interact with the cluster
  \end{itemize}
  \note<1>[item]{Number for recommended nodes from Apache}
  \note<1>[item]{ECC RAM and SSD are recommended}
  \note<1>[item]{DataStax recommends keeping data per node near or below 1 TB}
  % https://docs.datastax.com/en/dse-planning/doc/planning/planningHardware.html
  \note<2>[item]{Cloud is perfectly suited for this}
  \note<2>[item]{Two nodes cannot share a broadcast address (e.g. NAT)}
\end{frame}

\begin{frame}[fragile]{How to set up - Initial node}
/etc/cassandra/cassandra.yaml
\begin{semiverbatim}
# Opens socket on this address
listen_address: "192.168.0.2"
# Tells other nodes it's reachable on this address
broadcast_address: "3.14.1.59"

seed_provider:
  - class_name: org.apache.cassandra.locator.SimpleSeedProvider
    parameters:
      - seeds "3.14.1.59"

# Enable client communication
start_native_transport: true
\end{semiverbatim}
\note[item]{Own broadcast address is in seeds \rightarrow Creates own cluster}
\note[item]{For single node cluster use \lstinline{localhost} everywhere}
\end{frame}

\begin{frame}[fragile]{How to set up - Further nodes}
  /etc/cassandra/cassandra.yaml
  \begin{semiverbatim}
# Opens socket on this address
listen_address: "192.168.1.2"
# Tells other nodes it's reachable on this address
broadcast_address: "6.28.3.18"

seed_provider:
  - class_name: org.apache.cassandra.locator.SimpleSeedProvider
    parameters:
      - seeds "3.14.1.59"

# Enable client communication
start_native_transport: true
  \end{semiverbatim}
  \note[item]{Seeds is a comma separated "list" of IPs}
  \note[item]{Three per datacenter recommended}
\end{frame}

\begin{frame}{More configuration}
  \begin{itemize}
    \item Tune JVM parameters like heap size in \lstinline{cassandra-env.sh}
    \item Increase the number of tokens (virtual nodes) with \lstinline{num_tokens}
    \item Other performance, architecture or security tuning
  \end{itemize}
\end{frame}

\begin{frame}{Security}
  \begin{itemize}
    \item<1-> Use a firewall to not expose it to the public internet
    \item<2-> Internode communication with TLS
    \item<2-> Client $\leftrightarrow$ Node communication with TLS
    \item<3-> JMX management only on localhost and with auth
    \item<4-> Authentication with password
    \item<5-> Disable default role
    \item<5-> Use roles
  \end{itemize}
  \note<1>[item]{Basic linux due diligence}
  \note<2>[item]{TLS is always good \rightarrow Encryption, Authenticity}
  \note<5>[item]{There are no users, only roles (good RBAC model)}
\end{frame}

\subsection{Things to keep in mind}
% How to set up as an admin / Must knows

\section{Summary and Conclusion}

\end{document}
